Initially, we consider a somewhat naïve modelling of the golden eagle data.\cite{eagleData} The data contains $45.914$ observations of spatiotemporal information for $58$ total tracks divided into $599$ segments. We will use the variables: \texttt{Segment\_ID}, \texttt{Longitude}, \texttt{Latitude}, \texttt{Altitude}, \texttt{horizontal\_steps}. Consecutive observations are collected at one minute intervals. Although, the data has been tidied already, we will still do some minor wrangling; we calculate the difference in altitude between consecutive observations within segments. Obviously, by doing so we have to discard the first observation for each segment. This leaves us with $45.315$ observations. Apart from this there are no issues with the data, we depict simple information about the data in the appendix. Using the coordinates we depict the segments with coordinates in the north-eastern United States; the vast-majority of the data.
\begin{figure}[h]
    \begin{center}
        \includegraphics[scale = .085]{figures/allBirdTracks.jpeg}
        \caption{Longitude and latitude for the $44.730$ observations belonging to segments in the US.}
    \end{center}
\end{figure}
\subsection{Fitting an HMM to the Golden eagle tracks}
Having dealt with the initial 
\subsection{Simulating hidden markov models}

\subsection{Sampling correlated variables}\label{correlatedVariables}
Now, we deviate from the regular hidden markov model by allowing correlation in the observed chain. We have independently developed a method of simulating correlated variables with any parameterized distribution in a similar manner to \cite{thomasWard}. In later applications we will use the weibull and gaussian distributions, but the following sampling procedure applies to any imaginable parameterized distribution.\\ Firstly, sample $(X_1, X_2)$ from a bivariate gaussian distribution with mean vector, $\xi = \mathbf{0}$ and covariance matrix 
\begin{align}\Sigma = \begin{pmatrix}
    1 & \rho \\
    \rho & 1
\end{pmatrix}
\end{align}
By this construction $\rho$ is the correlation between $X_1$ and $X_2$. Now, we transform the samples in the following way: First transform the vector into bivariate uniform samples $Y = (\Phi(X_1),\Phi(X_2))$. Then let $Z = (G_1(Y_1), G_2(Y))$, where $G_1, G_2$ are quantile functions of parameterized distributions and not necessarily the same. By applying the quantile transformation theorem twice in a back and forth manner, we see that $Z_1$ and $Z_2$ has distributions corresponding to $G_1, G_2$ respectively. Obviously the correlation is no longer $\rho$ since the maps $\Phi$ and $G_1, G_2$ are non-linear. As finding an analytic expression for it is a very diffucult task, we will let it be up to experimentation to find out what the new correlation exactly is. However, as we will see, sampling this way almost always lead to some correlation, and we can somewhat control the sign and magnitude of it. And, most importantly, we can completely pick marginal distributions ourselves. Additionaly, due to the distributions being parameterized, we can just let the parameters be dependent on the value of the underlying state-sequence. Finding the density however can be quite cumbersome. We find the density of the weibull-normal distribution in appendix \ref{weibullGaussianAppendix}.