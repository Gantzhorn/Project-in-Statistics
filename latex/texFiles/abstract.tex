In this project we use Hidden markov models (HMM) to model three dimensional movements of golden eagles \cite{eagleData}. We investigate the working assumption of conditional independence of the observations given the underlying states by means of fitting a basic HMM to the eagle data and then use it to do a simulation study examining the consequences of the violation of the assumption. More specifically, we derive a bivariate weibull-normal distribution in which we can control the correlation and use it to demonstrate that assuming conditional independence in the model when there are large correlations in the data has negative consequences for the quality of the estimates. Finally, we use the derived weibull-normal distribution to extend the basic HMM for the 3D-movement of the eagles and compare our findings to other projects that have looked into the same or similar data. 