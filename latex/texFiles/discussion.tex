We finalize our work with HMM's by a brief look into a bit of model selection and possible extenstions to the model. Additionaly, we compare our results to previous work. 
\subsection{Selecting a model and possible extensions}
We start of the model selection by noting that the biggest difference in the two models are the MLE for the observed state sequence. Confer table \ref{changeEstimFit} in appendix \ref{auxiliaryResultsModelling} most observations have had consistent labelling, and it is only similar states that have switched labels i.e. the very dissimilar states \textit{Gliding} and \textit{Perching} have not had any switching. The QQ-plot on figure \ref{combinedQQPlots} viz-a-viz figure \ref{combinedQQPlotsFinal} seem to favor the more complex model slightly. On the contrary, there was generally a significant autocorrelation in the data not captured by either model; it is possible to extend the model with a second order markov chain \cite{Thede1999} i.e. a model where the distribution of the chain at time $t$ also depends on the value at time $t-2$.\\
However for only three states, the results are quite promising for the legitimacy of using HMMs in this context. The motivation for selecting only three states was mostly easier interpretability of each of the states. A biologist or an expert would undoubtly find a better fit with more states while via their expertise preserving interpretable states. On the same data $5$ states has been used \cite{Pirotta2018} for instance. Naturally, it is difficult to capture such a complex phenomenon as 3- dimensional movement of golden eagles with only three behavioural states. Still, if we allow ourselves to interpret the names of the hidden states a bit liberaly; the results in tables \ref{estimWeight1} and \ref{estimWeight2} seem comparable with (table 1) in \cite{Pirotta2018}. As we did not have covariates in the model, we compare with the \textit{overall} row: Their results in terms of how often the birds were perching are quite similar to ours - around $10\%$ of the observation are in this state. On the other hand, the sum of their soaring-states constituted around $60\%$ of the observations, whereas their gliding states were around $30\%$. Recall that we found a total share of the of roughly $50\%$ and $40\%$ respectively. Of course it is inherently difficult to compare 3 states to 5 even when the naming might encourage it. For instance their "orographic soaring" \textit{"Showed large variation in the vertical drift (ed. vertical steps), suggesting irregular gaining and losing of altitude."} \cite{Pirotta2018} (p. 2010) - this description seems to resemble a combination of or soaring- and gliding states that often saw either large increases or decreases in altitude respectively (see figure \ref{bird539}). The findings of significant persistence in states, as demonstrated in both equations (\ref{gammaMat2}) and \ref{gammaMat1}, align well with our intution about behaviours of animals, and, more importantly, existing research in the field see for instance \cite{UncoveringEcologicalState}. However, the pronounced persistence of states according to (\ref{gammaMat2}) diverges a bit from these results, suggesting a stronger alignment with our initial model.
Additionally, it is worth noting that numerous studies in the field, for instance, \cite{Pirotta2018}, incorporate angles in their models, a component which was not included in our analysis. Considering its widespread use and the fact that it actually is readily available in the data \cite{eagleData}, the inclusion of this variable presents an obvious way of enhancing the model's efficacy.
Yet another viable avenue of exploration to refine the current model, could be to extend the it to a non-parametric HMM. Numerous variants of non-parametric HMMs have been developed; for instance HMM based on P-spliens \cite{NonparametricHMM}, and their utility in animal movement modelling has gained increasing recognition in recent research amongst others in \cite{Cullen2021} (and accompanying R-package \cite{bayesmove}). This suggests another direction for further improving the model. Also, non-parametric models allows us to skip selecting parameterized distributions that can ultimately hamper the flexibility of our model.\\
This exactly is a problem with our extension of the basic HMM. A key issue we have not considered yet is if the dependence structure of the weibull-normal distribution, even remotely matches that of the data. While we did not derive the exact dependence structure, figures \ref{correlationEagleDensities} and \ref{correlationDensities1} suggest a relatively linear dependency. However, it's important to note that the monte-carlo estimate of the linear-correlation may not completely capture the intricacy of the, perhaps non-linear, dependence structure present in the data. On the other hand, this issue might be fixed if we allowed for each of the states to have their own correlation i.e. $\rho$ was state dependent. Yet that is also a topic for further research. What we did manage to find and demonstrate in figure \ref{logbiasPlot1}, is that some violations of the conditional independence assumption lead to significant bias in the estimated parameters.