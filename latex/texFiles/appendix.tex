\section{Summary statistics and marginal distributions for the golden eagle data}\label{summarStatistics}
\begin{table}[ht]
    \centering
    \begin{tabular}{lllll}
      \hline
    \textbf{Statistic} &   \textbf{Longitude} &    \textbf{Latitude} & \textbf{horizontal\_steps} & \textbf{vertical\_steps} \\ 
      \hline
    Min. &   -80.47   & 38.91   &    0.226   & -2285.2101   \\ 
      1st Quantile & -78.32   & 40.22   & 233.566   & -55.4587   \\ 
      Median & -77.56   & 41.06   &  519.348   &    -0.3183   \\ 
      Mean   & -77.31   & 41.00   & 566.544   & -0.9842   \\ 
      3rd Quantile & -76.47   & 41.64   & 855.376   & 51.3704   \\ 
      Max.   & -69.43   & 48.52   & 3807.510   & 996.1946   \\ 
       \hline
    \end{tabular}
    \caption{Short statistics from R's summary command for the most important variables in the eagle data}
    \label{tabularEagles}
\end{table}
\begin{figure}[h!]
    \centering
        \includegraphics[scale = .175]{figures/horizontalAndVerticalDensities.jpeg}
        \caption{Marginal densities for step lengths in both dimensions and their joint distribution\\ Note that the axes depicting the horizontal steps are shown on a logarithmic scale}
        \label{densityEagles}
\end{figure}
\newpage
\section{Correlated data auxiliary results}\label{simWeibullGaussian}
\subsection{Simulation study}\label{simstudyappendix}
For varying levels of correlation $\rho$ in the bivariate normal we simulate $2.500$ gaussian weibull variables with $k=2, \; \lambda = 5, \; \mu = 4, \; \sigma = 3$. For each correlation we use MLE to estimate the parameters. Additionaly, we find the Monte Carlo estimate for the correlation of the weibull-gaussian data, $\hat{\rho}_{prev}$
\begin{table}[ht]
    \centering
    \begin{tabular}{rrrrrr}
          \hline
          $\rho$ & $\hat{k}$ & $\hat{\lambda}$ & $\hat{\mu}$ & $\hat{\sigma}$ & $\hat{\rho}_{prev}$\\ 
          \hline
        -0.9 & 2.016813 & 4.999784 & 4.0 & 3.0 & -0.886736 \\ 
          -0.45 & 1.990909 & 5.000908 & 4.0& 3.0 & -0.443799 \\ 
          -0.1 & 1.993609 & 4.999045 & 4.0 & 3.0 & -0.100261 \\ 
          0.9 & 2.000166 & 5.001516 & 4.0 & 3.0 & 0.889852 \\ 
          0.45 & 2.000223 & 5.001611 & 4.0 & 3.0 & 0.440810 \\ 
          0.1 & 2.015213 & 5.002888 & 4.0 & 3.0 & 0.101309 \\ 
           \hline
        \end{tabular}
    \caption{Estimated parameters the simulated data}
    \label{correlationTable}
\end{table}\\
We plot the joint - and marginal distributions for the data in table \ref{correlationTable}.
\begin{figure}[h!]
        \begin{center}
        \includegraphics[scale = .225]{figures/correlatedWeibullNormalSim.jpeg}
        \caption{Joint - and marginal distributions for the data in table \ref{correlationTable} with negative correlation in the first row and positive in the second row}
    \end{center}
    \label{correlationDensities}
\end{figure}

\subsection{Imitated eagle data}
This section shows analagously to appendix \ref{simstudyappendix} the results of altering the correlation in the eagle data. This was done like explained in section \ref{eagleImitation}. Using the same values as in appendix \ref{simstudyappendix}. The sampling resulted in the monte-carlo estimates, $\hat{\rho}$ for the correlation of $\rho$.
\begin{table}[ht]
    \centering
    \begin{tabular}{rrr}
      \hline
     & $\rho$ & $\hat{\rho}$ \\ 
      \hline
    1 & -0.900000 & -0.706768 \\ 
      2 & -0.450000 & -0.458117 \\ 
      3 & -0.100000 & -0.276326 \\ 
      4 & 0.900000 & 0.269978 \\ 
      5 & 0.450000 & 0.024760 \\ 
      6 & 0.100000 & -0.167893 \\ 
       \hline
    \end{tabular}
    \caption{Estimated correlation in the altered eagle data based on $\rho$.}
    \label{correlatedEagleTable}
    \end{table}

    \begin{figure}[h!]
        \begin{center}
        \includegraphics[scale = .15]{figures/correlatedEaglesSim.jpeg}
        \caption{Joint - and marginal distributions for the data in table \ref{correlatedEagleTable} with negative correlation in the first row and positive in the second row}
    \end{center}
    \label{correlationEagleDensities}
\end{figure}
\newpage
\section{Findings auxilliary results}\label{auxiliaryResultsModelling}
This part shows different additional plots and graphs found in the modelling and simulations. 
\subsection{Fitting an HMM to the Golden eagle segments}\label{goldenEagleSegments}
\newpage
\section{Finding the density of the Weibull-Gaussian distribution}\label{weibullGaussianAppendix}
To find the density of the Weibull-Gaussian distribution, we let $(X_1, X_2)$ be bivariate gaussian with mean vector, $\xi = \mathbf{0}$ and covariance matrix $\Sigma = \begin{pmatrix}
    1 & \rho \\
    \rho & 1
\end{pmatrix}$. Now, we apply the methodology described in section \ref{correlatedVariables} in the following way. Let $Y$ be the transformed variable. Since we want $Y_1$ just to be a non-standard gaussian we can skip the quantile tranformation theorem on this entry. Therefore define the map $h_1: \mathbb{R}^2 \mapsto \mathbb{R}$ by $h_1(x_1, x_2) = x_1\sigma + \mu$ and $h_2:\mathbb{R}^2 \mapsto \mathbb{R}$ as $ h_2(x_1, x_2) = \lambda\left(-\log\left(1-\Phi(x_2)\right)\right)^{\frac{1}{k}}$. Defining 
\begin{align}
    Y = \begin{pmatrix}
        h_1(x_1, x_2)\\
        h_2(x_1, x_2)
    \end{pmatrix}    
\end{align}
we have be the quantile tranformation theorem that
$Y_1 \sim \mathcal{N}(\mu, \sigma^2)$ and $Y_2 \sim Wei(k, \lambda)$. Recall that the multivariate density transformation theorem states that $Y$ then has density
\begin{align}
    g(y_1, y_2) = f\left(h_1^{-1}(y_1, y_2), h_2^{-1}(y_1, y_2)\right)\abs{J(y_1, y_2)} \label{gDensity}
\end{align}
Where $f$ is the density of the bivariate normal distribution
\begin{align}
    f(\mathbf{x}) = \frac{1}{2\pi\sqrt{\det(\Sigma)}}\exp\left(-\frac{1}{2}\mathbf{x}^\top \Sigma^{-1}\mathbf{x}\right)
\end{align}
In (\ref{gDensity}) the jacobian $J$ is 
\begin{align}
    J(y_1, y_2) = \det\left(\begin{matrix}
        \frac{\partial h_1^{-1}(y_1, y_2)}{\partial y_1} & \frac{\partial h_1^{-1}(y_1, y_2)}{\partial y_2} \\
        \frac{\partial h_2^{-1}(y_1, y_2)}{\partial y_1} & \frac{\partial h_2^{-1}(y_1, y_2)}{\partial y_2}
    \end{matrix}\right)  
\end{align}
We easily find
\begin{align}
    h_1^{-1}(y_1, y_2) = \frac{y_1-\mu}{\sigma}, \qquad 
    h_2^{-1}(y_1, y_2) = \Phi^{-1}\left(1-e^{-\left(\frac{y_2}{\lambda}\right)^k}\right) 
\end{align}
Differentiating and setting into the definition of the jacobian yields:
\begin{align}
    J(y_1, y_2) = \frac{k\left(\frac{y_2}{\lambda}\right)^ke^{-\left(\frac{y_2}{\lambda}\right)^k}}{\varphi\left(\Phi^{-1}\left(1-e^{-\left(\frac{y_2}{\lambda}\right)^k}\right)\right)y_2\sigma }    
\end{align}
where $\varphi$ is the density of the standard normal distribution. We can then obtain the density by combining above formulae.
\newpage
\section{Benchmark for various implementations}
In the section we show the results of various benchmarks using \cite{microbenchmark}. The implementations were run with the following hardware and software.
\begin{table}[ht]
    \centering
    \begin{tabular}{ll}
      \hline
      \textbf{Specification} & \textbf{Description} \\
      \hline
      Processor (CPU) & Intel i7-4800MQ (8) @ 3.700GHz \\
      RAM & 16 GB \\
      Operating System & Linux Mint 21.1 x86\_64 \\
      Storage & 500 GB SSD \\
      Graphics Card (GPU) & Intel 4th Gen Core Processor \\
      Compiler and Software & R 4.3.0 with C++ compiler g++ 11.3.0\\
      \hline
    \end{tabular}
  \end{table}
  
\subsection{Markov chain sampler}
We compare the two markov chain samplers
\begin{figure}[h!]
    \centering
    \includegraphics[scale = .15]{figures/markovChainSamplers.jpeg}
    \caption{The basic R markov chain sampler compared to more sophisticated C++ implementation}
    \label{markovChainRvRcpp}
\end{figure} 