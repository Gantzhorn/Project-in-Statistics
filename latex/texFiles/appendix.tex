\section{Finding the density of the Weibull-Gaussian distribution}\label{weibullGaussianAppendix}
To find the density of the Weibull-Gaussian distribution, we let $(X_1, X_2)$ be bivariate gaussian with mean vector, $\xi = \mathbf{0}$ and covariance matrix $\Sigma = \begin{pmatrix}
    1 & \rho \\
    \rho & 1
\end{pmatrix}$. Now, we apply the methodology described in section \ref{correlatedVariables} in the following way. Let $Y$ be the transformed variable. Since we want $Y_1$ just to be a non-standard gaussian we can skip the quantile tranformation theorem on this entry. Therefore define the map $h_1: \mathbb{R}^2 \mapsto \mathbb{R}$ by $h_1(x_1, x_2) = x_1\sigma + \mu$ and $h_2:\mathbb{R}^2 \mapsto \mathbb{R}$ as $ h_2(x_1, x_2) = \lambda\left(-\log\left(1-\Phi(x_2)\right)\right)^{\frac{1}{k}}$. Defining 
\begin{align}
    Y = \begin{pmatrix}
        h_1(x_1, x_2)\\
        h_2(x_1, x_2)
    \end{pmatrix}    
\end{align}
we have be the quantile tranformation theorem that
$Y_1 \sim \mathcal{N}(\mu, \sigma^2)$ and $Y_2 \sim Wei(k, \lambda)$. Recall that the multivariate density transformation theorem states that $Y$ then has density
\begin{align}
    g(y_1, y_2) = f\left(h_1^{-1}(y_1, y_2), h_2^{-1}(y_1, y_2)\right)\abs{J(y_1, y_2)} \label{gDensity}
\end{align}
Where $f$ is the density of the bivariate normal distribution
\begin{align}
    f(\mathbf{x}) = \frac{1}{2\pi\sqrt{\det(\Sigma)}}\exp\left(-\frac{1}{2}\mathbf{x}^\top \Sigma^{-1}\mathbf{x}\right)
\end{align}
In (\ref{gDensity}) the jacobian $J$ is 
\begin{align}
    J(y_1, y_2) = \det\left(\begin{matrix}
        \frac{\partial h_1^{-1}(y_1, y_2)}{\partial y_1} & \frac{\partial h_1^{-1}(y_1, y_2)}{\partial y_2} \\
        \frac{\partial h_2^{-1}(y_1, y_2)}{\partial y_1} & \frac{\partial h_2^{-1}(y_1, y_2)}{\partial y_2}
    \end{matrix}\right)  
\end{align}
We easily find
\begin{align}
    h_1^{-1}(y_1, y_2) = \frac{y_1-\mu}{\sigma}, \qquad 
    h_2^{-1}(y_1, y_2) = \Phi^{-1}\left(1-e^{-\left(\frac{y_2}{\lambda}\right)^k}\right) 
\end{align}
Differentiating and setting into the definition of the jacobian yields:
\begin{align}
    J(y_1, y_2) = \frac{k\left(\frac{y_2}{\lambda}\right)^ke^{-\left(\frac{y_2}{\lambda}\right)^k}}{\varphi\left(\Phi^{-1}\left(1-e^{-\left(\frac{y_2}{\lambda}\right)^k}\right)\right)y_2\sigma }    
\end{align}
where $\varphi$ is the density of the standard normal distribution. We can then obtain the density by combining above formulae.