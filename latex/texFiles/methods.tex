We primarily implement and find our results using the R langauge.
Not only do we implement our own software, the package \code{momentuHMM} \cite{momentuHMM} is a paramount part the modelling process itself. However, we first introduce underlying model and its assumptions.
\subsection*{Model formulation}
The two stochastics processes from the Hidden Markov Model are: the state process, $S_t$ and the observation $\mathbf{Y_t}$, which are latent and observed respectively. The variation of the HMM we work with is a discrete-time and finite state HMM i.e. $t\in\{1,\dots , M\}$ and $S_t\in\{1,\dots , N\}$ for some $M, N\in\mathbb{N}$. We assume that the state space is determined by an initial probability vector, $\delta = \left(\delta_1,\dots, \delta_N \right)$. Furthermore, it has transition probability matrix
\[
    \Gamma = \begin{pmatrix}
        \gamma_{11} & \dots &  \gamma_{1N} \\
        \vdots & \ddots & \vdots \\
        \gamma_{N1} & \dots & \gamma_{NN}
    \end{pmatrix}
\]
where $\gamma_{ij}$ is the probability of jumping from state $i$ to state $j$. We assume that the chain has the markov property. That is $\gamma_{ij} = P\left(S_{t+1} = j | S_t = i\right)$. Now, the observed process depends on the value of the state process in that it is drawn from some state-dependent distribution. We assume that two values $\mathbf{Y_i}, \mathbf{Y_j},\; i\neq j$ are conditionally independent given the state process.