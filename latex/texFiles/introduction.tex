We consider the class of models called discrete-time hidden Markov models (HMM); a popular model for doing inference on animal movement based on telemetry data such as measurements of latitude, longitudes and metres above mean sea level (MAMSL) at fixed time intervals. This data is then often used to calculate the distance an animal has travelledbetween measurements or a similar metric such as the angle formed by the last three measurements.\\ Mathematically, HMMs consists of two stochastic processes: The state process, also called the hidden- or latent process, and the observed- or state-dependent process. 
We use the distribution of the observed process to model the measurements from our telemetry data. The main idea in hidden markov models is that these measurements somehow depend on the underlying state process, and thus can serve as a proxy of it. When we model animal movements the state process could for instance be correspond to the various, for us unknown, actions an animal does e.g.: Resting, flying and eating. \cite{HHMForTimesSeries} We can draw the dependence structure of HMMs as a graphical model
\begin{figure}[h]
    \centering
    \includegraphics[scale = .8]{figures/tikzPictures/hiddenMarkovChain.pdf}
    \caption{A hidden markov model with state process: $S$ and state-dependent process: $\mathbf{Y}$.}
\end{figure}\\
Many would agree that the raison d'être of HMM's is that the infered hidden states allow for easy interpretation. With regards to animal movement modelling: Observing an animal directly can be impractical or nigh impossible. However, we can for various reasons still be interested in the behaviour of individuals or a population of animals. In such cases, HMM allows us to do this inference via the hidden states.  Another advantage of the model is that there exists efficient algorithms for the fitting. Another assumption used extensively in the field of HMMs is contemporaneous conditional independence in the observed state vector given the value of the hidden state. An immediate consequence of this is that we would be equally likely to see large changes in altitudes regardless of the change in horizontal positon of the animal. Intuitively though, we can easily imagine that this assumption is violated by real animals. Due to energy budgets an eagle is unlikely to fly long horizontal distances, if it also gains altitude aplenty. \\
We will investigate this assumption and what consequences the violation of it has for the model fitting. Finally, we look into one way of extending the model to relax this assumption.