We consider the class of models called discrete-time hidden Markov models (HMM); a popular model for doing inference on animal movement based on telemetry data such as measurements of latitude, longitudes and metres above mean sea level (MAMSL) at fixed time intervals. This data is then often used to calculate the distance an animal has travelled since the between measurements or a similar metric such as the angle formed by the last three measurements.\\ Mathematically, HMMs consists of two stochastic processes: The state process also called the hidden- or latent process and the observed- or state-dependent process. 
We use the distribution of the observed process to model the measurements from our telemetry data. The idea is that these measurements depend on the underlying state process and thus can serve as a proxy of it. When we model animal movements the state process would for instance be equivalent to the various, to us unobserved, actions an animal could do e.g.: Resting, flying and eating. We can draw this model as a graphical model
\begin{figure}[h]
    \centering
    \includegraphics[scale = .8]{figures/tikzPictures/hiddenMarkovChain.pdf}
    \caption{A hidden markov model with state process: $S$ and state-dependent process: $\mathbf{Y}$.}
\end{figure}\\
Many would agree that the raison d'être of HMM's is that the infered hidden states are quite easily interpreted. In particular with regards to animal movement modelling, in these models we are namely often interested in doing inference on the behaviour of the animal in question via the hidden states. Another advantage of the model is that there exists efficient algorithms for the fitting, utilizing the so-called forward algorithm. However, this hinges on a central assumption of independence in the observed sequence conditional on the state. An assumption used extensively in the field of HMMs, which one obviously has to investigate. Another central assumption is the independence between the observations in the observed state vector at a given time $t$. Intuitively though, one could easily imagine that this assumption cannot hold for real animals. Due to energy budgets it is difficult to imagine an eagle being able to fly long horizontal distances if it also gains altitude aplenty. We will investigate this assumption and what consequences the violation of it has for the model fitting. Finally, we will see how we can extend the model to relax this assumption.